%% This template can be used to write a paper for
%% Computer Physics Communications using LaTeX.
%% For authors who want to write a computer program description,
%% an example Program Summary is included that only has to be
%% completed and which will give the correct layout in the
%% preprint and the journal.
%% The `elsarticle' style is used and more information on this style
%% can be found at 
%% http://www.elsevier.com/wps/find/authorsview.authors/elsarticle.
%%
%%
\documentclass[preprint,12pt]{elsarticle}

%% Use the option review to obtain double line spacing
%% \documentclass[preprint,review,12pt]{elsarticle}

%% Use the options 1p,twocolumn; 3p; 3p,twocolumn; 5p; or 5p,twocolumn
%% for a journal layout:
%% \documentclass[final,1p,times]{elsarticle}
%% \documentclass[final,1p,times,twocolumn]{elsarticle}
%% \documentclass[final,3p,times]{elsarticle}
%% \documentclass[final,3p,times,twocolumn]{elsarticle}
%% \documentclass[final,5p,times]{elsarticle}
%% \documentclass[final,5p,times,twocolumn]{elsarticle}

%% if you use PostScript figures in your article
%% use the graphics package for simple commands
%% \usepackage{graphics}
%% or use the graphicx package for more complicated commands
%% \usepackage{graphicx}
%% or use the epsfig package if you prefer to use the old commands
%% \usepackage{epsfig}

%% The amssymb package provides various useful mathematical symbols
\usepackage{amssymb}
%% The amsthm package provides extended theorem environments
%% \usepackage{amsthm}

%% The lineno packages adds line numbers. Start line numbering with
%% \begin{linenumbers}, end it with \end{linenumbers}. Or switch it on
%% for the whole article with \linenumbers after \end{frontmatter}.
%% \usepackage{lineno}

%% natbib.sty is loaded by default. However, natbib options can be
%% provided with \biboptions{...} command. Following options are
%% valid:

%%   round  -  round parentheses are used (default)
%%   square -  square brackets are used   [option]
%%   curly  -  curly braces are used      {option}
%%   angle  -  angle brackets are used    <option>
%%   semicolon  -  multiple citations separated by semi-colon
%%   colon  - same as semicolon, an earlier confusion
%%   comma  -  separated by comma
%%   numbers-  selects numerical citations
%%   super  -  numerical citations as superscripts
%%   sort   -  sorts multiple citations according to order in ref. list
%%   sort&compress   -  like sort, but also compresses numerical citations
%%   compress - compresses without sorting
%%
%% \biboptions{comma,round}

% \biboptions{}

%% This list environment is used for the references in the
%% Program Summary
%%
\usepackage{comment}
\usepackage{braket}
\usepackage{mathrsfs}
\newtheorem{definition}{Definition}[section]
\usepackage{mathtools}
\usepackage{indentfirst}
\usepackage{pgfplots}
\usepackage{physics}
\DeclarePairedDelimiter\ceil{\lceil}{\rceil}
\DeclarePairedDelimiter\floor{\lfloor}{\rfloor}
\usepackage{subfiles}
\usepackage{caption}
\usepackage{subcaption}
\usepackage{graphicx}
\usepackage{hyperref}
\usepackage{rotating}
\usepackage{lipsum}
\usepackage{blindtext}
\usepackage{listings}
\usepackage{xcolor}
\usepackage{myStyle}
\usepackage{adjustbox}
\renewcommand\ttdefault{cmvtt} 
\newcommand{\mysinglefigurescale}{0.35}

\journal{Computer Physics Communications}

\begin{document}

\begin{frontmatter}

%% Title, authors and addresses

%% use the tnoteref command within \title for footnotes;
%% use the tnotetext command for the associated footnote;
%% use the fnref command within \author or \address for footnotes;
%% use the fntext command for the associated footnote;
%% use the corref command within \author for corresponding author footnotes;
%% use the cortext command for the associated footnote;
%% use the ead command for the email address,
%% and the form \ead[url] for the home page:
%%
%% \title{Title\tnoteref{label1}}
%% \tnotetext[label1]{}
%% \author{Name\corref{cor1}\fnref{label2}}
%% \ead{email address}
%% \ead[url]{home page}
%% \fntext[label2]{}
%% \cortext[cor1]{}
%% \address{Address\fnref{label3}}
%% \fntext[label3]{}

\title{QWAK: Quantum Walk Analysis Kit}

%% use optional labels to link authors explicitly to addresses:
%% \author[label1,label2]{<author name>}
%% \address[label1]{<address>}
%% \address[label2]{<address>}

\author[a]{Jaime Santos\corref{author}}
\author[b]{Bruno Chagas}
\author[c]{Rodrigo Chaves}
\author[d]{Lorenzo Buffoni}

\cortext[author] {Corresponding author.\\\textit{E-mail address:} jaimepereirasantos123@gmail.com}
\address[a]{Haslab, INESC TEC, Portugal}
\address[b]{Mastercard, Dublin, Ireland}
\address[c]{Universidade Federal de Minas Gerais, Brazil}
\address[d]{Department of Physics and Astronomy, University of Florence, 50019 Sesto Fiorentino, Italy}

\begin{abstract}
In this paper, we describe a continuous-time quantum walk (CTQW) simulation
package for \texttt{Python 3}, covering their theoretical foundations and
practical applications. The software provides both unitary and open system
evolution of over general graphs, alongside tools for visualization and
exploration of several different aspects of the quantum walk.  We go over
installation, design and performance of the package, concluding with several
examples on how \texttt{QWAK} can be used to explore problems such as search,
perfect state transfer, among others.
\end{abstract}


\begin{keyword}
% keywords here, in the form: keyword \sep keyword
Quantum Computing; Continuous-Time Quantum Walks; Stochastic Quantum Walks; Transport Properties

\end{keyword}

\end{frontmatter}

%%
%% Start line numbering here if you want
%%
% \linenumbers

% All CPiP articles must contain the following
% PROGRAM SUMMARY.

{\bf PROGRAM SUMMARY/NEW VERSION PROGRAM SUMMARY}
  %Delete as appropriate.

\begin{small}
\noindent
{\em Program Title: }QWAK                                          \\
{\em CPC Library link to program files:} (to be added by Technical Editor) \\
{\em Developer's repository link: } \url{https://github.com/JaimePSantos/QWAK}\\
{\em Code Ocean capsule:} (to be added by Technical Editor)\\
{\em Licensing provisions:} CC By 4.0 \\
{\em Programming language: } Python, Javascript, HTML and CSS.                                  \\
% {\em Supplementary material:}                                 \\
  % Fill in if necessary, otherwise leave out.
% {\em Journal reference of previous version:}*                  \\
  %Only required for a New Version summary, otherwise leave out.
% {\em Does the new version supersede the previous version?:}*   \\
  %Only required for a New Version summary, otherwise leave out.
% {\em Reasons for the new version:*}\\
  %Only required for a New Version summary, otherwise leave out.
% {\em Summary of revisions:}*\\
  %Only required for a New Version summary, otherwise leave out.
{\em Nature of problem:}
  \texttt{QWAK} simulates unitary and stochastic continuous-time quantum walks,
  focusing on transport property analysis, applications, visualization and ease
  of use.\\
{\em Solution method:}
 We leverage \texttt{Python}'s vast package resources such as \texttt{NumPy}
 and \texttt{NetworkX} to implement the desired structures, and then generate
 the Hamiltonians via spectral decomposition. For the stochastic case, the
 Lindblad master equations are solved with \texttt{Qutip}.\\
{\em Additional comments including restrictions and unusual features:}\\
  %Provide any additional comments here.
  The GUI provided uses \texttt{MongoDB} to store \texttt{QWAK} objects, which
  currently limits the size of the graphs since the adjacency matrices are also
  stored.  \\
\end{small}

\section{Introduction}
    \subfile{intro}


\section{Theoretical Background}\label{sec:theor_background}
    \subfile{TheorBackgroundIntro}
    \subsection{Continuous-Time Quantum Walk}
        \subfile{contTimeQW}
    \subsection{Transport Properties}
        \subfile{transportProp-intro}
        \subfile{transportProp}
    \subsection{Searching Algorithm}
        \subfile{SearchingAlgorithm}\label{sec:theor_searching}
    \subsection{Stochastic Quantum Walk}
        \subfile{stochasticCtqw}

\section{Package Overview}\label{sec:pack_overview}
    \subfile{qwak-intro}
    \subsection{Istallation and dependencies}
        \subfile{installation}
    \subsection{Package Description}
        \subfile{overview}\label{sec:OverviewDescription}
    \subsection{Graphical Interface}
        \subfile{graphicalInterface}

\section{Use Cases}\label{sec:use_cases}
    \subfile{qwak-usage-intro}
    \subsection{Perfect State Transfer}
        \subfile{perfectStateTransfer}
    \subsection{Oriented Quantum Walk}
        \subfile{orientedQuantumWalk}
    \subsection{Searching with QWAK}
        \subfile{searchingQuantumWalk}
    \subsection{Stochastic QWAK on a Maze}
        \subfile{stochasticQuantumWalk}


\section{Conclusion}\label{sec:conclusion}
    \subfile{conclusion}

\clearpage

%% main text
\section*{Funding}
This work is financed by National Funds through the Portuguese funding agency,
FCT - Fundação para a Ciência e a Tecnologia, within project UIDB/50014/2020 and project 2023.02269.BD. 

This work was partially funded  through FCT - Fundação para a Ciência e a Tecnologia, I.P. (Portuguese Foundation for Science and Technology) within the project IBEX, with reference 10.54499/PTDC/CCI-COM/4280/2021.

\sloppy
L.B. received funding from Next Generation EU, in the context of the National
Recovery and Resilience Plan, M4C2 investment 1.2. Project SOE0000098-ThermoQT. This resource was financed by the Next Generation EU [DD 247 19.08.2022]. The
views and opinions expressed are only those of the authors and do not
necessarily reflect those of the European Union or the European Commission.
Neither the European Union nor the European Commission can be held responsible
for them.

%% The Appendices part is started with the command \appendix;
%% appendix sections are then done as normal sections
%% \appendix

%% \section{}
%% \label{}

%% References
%%
%% Following citation commands can be used in the body text:
%% Usage of \cite is as follows:
%%   \cite{key}         ==>>  [#]
%%   \cite[chap. 2]{key} ==>> [#, chap. 2]
%%

%% References with bibTeX database:

\bibliographystyle{elsarticle-num}
\bibliography{bibliography}

%% Authors are advised to submit their bibtex database files. They are
%% requested to list a bibtex style file in the manuscript if they do
%% not want to use elsarticle-num.bst.

%% References without bibTeX database:

% \begin{thebibliography}{00}

%% \bibitem must have the following form:
%%   \bibitem{key}...
%%

% \bibitem{}

% \end{thebibliography}


\end{document}

%%
%% End of file 
