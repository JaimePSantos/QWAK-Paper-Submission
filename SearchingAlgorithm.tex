\documentclass[main.tex]{subfiles}
\begin{document}

Classically, searching an unstructured database involves exhaustively
evaluating each element. Given an oracle that checks element equality, a
specific element in a size \( N \) database would require, on average,
$\frac{N}{2}$ queries, and at most $N$ queries. Grover's algorithm
\cite{grover1996} is a quantum alternative that offers a quadratic speedup,
locating a target in $\mathcal{O}(\sqrt{N})$ time. However, direct
implementations of Grover's iteration on physical databases are challenging.
Notably, the CTQW can be modified to perform search \cite{farhi1996} and
address this issue, delivering similar speedups in broader structures.

The searching algorithm using the CTQW can be formulated by introducing a
search component into the initial Hamiltonian, typically derived from an
adjacency or Laplacian matrix. The search Hamiltonian is then formed by summing
projectors over a set of integers $M$, where indices mark the desired
element in the graph, and can be defined as

\begin{equation}
    H^\prime = -\gamma A - \sum_{x \in M} \ket{x}\bra{x},
    \label{eq:searchHamiltonian}
\end{equation}
where $\gamma \in \mathbb{R}$ is the critical transition rate, which must be
chosen for optimality and it is a graph-dependent parameter.

It is known that probability of order $1$ is reached in optimal time for a
myriad of different structures, such as the complete graph \cite{farhi1996},
the hypercube and $d$-dimensional periodic lattice \cite{childs2004}, and the
family of strongly regular graphs \cite{Janmark2014}. In fact, search by CTQW
is optimal for \textit{almost all} graphs \cite{chakraborty2016}, given the
correct conditions.\par

\end{document}


