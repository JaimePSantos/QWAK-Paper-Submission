\documentclass[main.tex]{subfiles}
\begin{document}

\subsubsection{Statistical Measures}

Transport properties of CTQWs can be quantified using metrics like mean and
standard deviation, which shed light on the walker's position tendencies and
propagations. The mean is defined as
\begin{equation}
    \mu = \sum_{x} x |\langle \psi | x \rangle|^2
\end{equation}
and the standard deviation is
\begin{equation}
    \sigma = \sqrt{\sum_{i} (x - \mu)^2 |\langle \psi | x \rangle|^2},
\end{equation}
notably, as aforementioned, the standard deviation evolves quadratically faster
than in the classical case \cite{grimmett2003}.

To quantify the walker's localization, we study the likelihood of finding the
walker at a specific position after time $t$, namely the \textit{survival
probability} \cite{Goenuelol2011}. For a symmetric position range of $ 
[k_0,k_1]$ , it is given by
\begin{equation}
    P_{[k_0,k_1]}(t) = \sum_{i=k_0}^{k_1} |\bra{i}\ket{\psi(t)}|^2.
\end{equation}

Moreover, the \textit{inverse participation ratio} (IPR) provides an estimate
of the average number of nodes over which the walker is uniformly spread, and
it is defined as
\begin{equation}
    \text{IPR(t)} = \frac{1}{\sum_{x} |\langle x | \psi(t) \rangle|^4},
\end{equation} 
with IPR nearing 1 for uniform spread across positions and approaching 0 when
the walker localizes on a single site as $N$ increases \cite{Danaci2021}.

\subsubsection{Perfect State Transfer}

Christandl \textit{et al.} \cite{christandlPerfect04} pioneered the exploration
of state transfers in graphs as a means of information transport. They
questioned whether an initial state at the beginning of a path graph could
relocate to its end after a certain time $t$. Their findings negated this for
undirected graphs, but their work inspired further investigations into which
graphs exhibit this phenomenon and under which conditions. This has since been
explored in both undirected \cite{Coutinho2014, Cheung2014, Zhou2014} and
directed graphs \cite{Lato2020, Cameron2014}.

As was previously done in equation \eqref{eq:contSimulUniOp}, given an
Hamiltonian associated to the adjacency matrix of a graph $G$, the time
evolution operator can be defined as $U(t) = e^{-itA}$.

Perfect state transfer (PST) occurs at time $\tau$ between vertices $a$ and $b$
iff
\begin{equation}
    \abs{\bra{a}U(\tau)\ket{b}}^2 = 1,
\end{equation}
such that $\ket{k}$ is the characteristic vector of the vertex $k$ (all entries
of the vector are zero except for the $k$-th position).

PST is not universal across all graphs; the adjacency matrix's eigenvalues and
eigenvectors must obey very specific criteria. The conditions are well
known for the undirected case \cite{Godsil2012}, but a complete description is
still an open problem for oriented graphs \cite{Lato2020}. Although finding a
graph with PST can be quite difficult, checking if the adjacency matrix, for
the undirected case, fits in the criteria can be done in polynomial time
\cite{Coutinho2017}. Examples of PST include cartesian products of graphs and
hypercube graphs \cite{Coutinho2014, Cheung2014, Zhou2014}. Throughout this
work, we'll showcase our algorithmic implementation that determines if a graph
exhibits PST and its corresponding time.

\end{document}
