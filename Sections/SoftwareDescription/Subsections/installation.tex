\documentclass[../../dissertation.tex]{subfiles}
\begin{document}

After installing the necessary dependencies , the \texttt{QWAK} package can be
installed with the following \texttt{pip} command:

\begin{lstlisting}[style=commands]
$ pip install qwak-sim
\end{lstlisting}

This software provides an easy-to-use set of tools for simulating the
probability distribution and investigating transport properties of
continuous-time quantum walks on various graphs. \texttt{NumPy} is essential
for its multidimensional arrays and high-performing functions, while
\texttt{SciPy} offers complementary functions such as matrix exponentiation and
inverse matrix calculation. \texttt{QuTiP} enables the simulation of open
quantum system dynamics.\par

\texttt{NetworkX} is responsible for the structures of our walks, offering a
wide range of pre-defined graphs and the ability to create custom graphs with
weights. It is also tasked with calculating the adjacency or Laplacian matrix
of the graph, depending on users inputs.\par 

For visualization, \texttt{Matplotlib} generates plots from simulated results.
Alternatively, users can visualize static and animated plots in the GUI, which
requires local installation of \texttt{Flask} and \texttt{PyMongo}. A web
application is also available on Heroku
\footnote{\url{https://qwak-sim.herokuapp.com/}}, where these packages serve as
the backend to handle user requests and manage data storage, enabling
communication between \texttt{QWAK} and the frontend.

\end{document}
