\documentclass[../../main.tex]{subfiles}
\begin{document}

Classically, searching an unstructured database involves exhaustively
evaluating each element. Given an oracle that checks element equality, a
specific element in a size \( N \) database would require, on average,
$\frac{N}{2}$ queries, and at most $N$ queries. Grover's algorithm
\cite{grover1996} is a quantum alternative that offers a quadratic speedup,
locating a target in $\mathcal{O}(\sqrt{N})$ time. However, direct
implementations of Grover's iteration on physical databases are challenging.
Notably, the CTQW can be modified to perform search \cite{farhi1996} and
address this issue, delivering similar speedups in broader structures.

The searching algorithm using the CTQW can be formulated by introducing a
search component into the initial Hamiltonian, typically derived from an
adjacency or Laplacian matrix. The search Hamiltonian is then formed by summing
projectors over a set of integers $M$, where indices mark the desired
element in the graph, and can be defined as
%We can formulate a searching algorithm using the the continuous-time quantum
%walk by introducing a search component into the initial Hamiltonian, which is
%established  using an adjacency or Laplacian matrix. Therefore, the search
%Hamiltonian is constructed by summing projectors across the set of integers $M$
%of integer number, with incides indicating the marked element within the graph.
%We have the Hamiltonian for the searching problem in graphs as

\begin{equation}
    H^\prime = -\gamma A - \sum_{x \in M} \ket{x}\bra{x},
    \label{eq:searchHamiltonian}
\end{equation}
where $\gamma \in \mathbb{R}$ is the critical transition rate, which must be
chosen for optimality and it is a graph-dependent parameter.
%where $\gamma \in \mathbb{R}$, critical jumping rate, must be chosen for
%optimality and it is a graph-dependent parameter.

It is known that probability of order $1$ is reached in optimal time for a
myriad of different structures, such as the complete graph \cite{farhi1996},
the hypercube and $d$-dimensional periodic lattice \cite{childs2004}, and the
family of strongly regular graphs \cite{Janmark2014}. In fact, search by CTQW
is optimal for \textit{almost all} graphs \cite{chakraborty2016}, given the
correct conditions.\par

%Searching through an unstructured database is a task classically achieved by
%exhaustively evaluating every element in the database. Assume there exists a
%black box (oracle) that can be asked to find out if two elements are equal.
%Since we're looking for a specific element in a database of size $N$, we'd have
%to query the oracle on average $\frac{N}{2}$ times or, in the worst case, $N$ times.\par
%
%Grover's algorithm \cite{grover1996} is a quantum alternative for unstructured
%searches, and it's popularity comes from it's quadratic speedup, being able to
%find a target element in time  time $\mathcal{O}(\sqrt{N})$. However,
%algorithms based on the Grover iteration, are not directly implementable over a
%physical database. As an alternative, it is known \cite{farhi1996} that the
%CTQW can provide a model to address this, resulting in comparable speedups in
%much more general structures.

% As was previously described, the continuous-time quantum walk model is defined
% by an evolution operator obtained by solving Schrödinger's equation

% \begin{equation}
% 	U(t) = e^{-iHt}.
% \end{equation}

% The search problem requires introducing an oracle to the Hamiltonian, which will
% mark an arbitrary vertex $m$ 

% \begin{equation}
% 	H' = -\gamma L - \sum_{m \in M}\ket{m}\bra{m},
%     \label{eq:searchHamiltonian}
% \end{equation}

% where $M$ is the set of marked vertices. The main challenge will then be
% finding the value of $\gamma$ that maximizes the probability associated wtih
% marked element $\ket{w}$ for the smallest time possible. 

