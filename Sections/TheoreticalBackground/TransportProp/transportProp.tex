\documentclass[../../main.tex]{subfiles}
\begin{document}

% It has been known for some time that the standard deviation of quantum walks scales
% faster in time than the classical random walk, for structures such as the cycle
% by \cite{aharonov2002}, the line with the work of \cite{ambainis2001}, and
% multi-dimensional lattices as shown by \cite{santos2015}. Quantum interference
% is responsible for the improvements offered by quantum walk transport, which
% have the potential to reach exponential speedups in applications such as
% \textit{hitting time} in hypercubic graphs as shown by  \cite{jkempe02}, and on
% glued trees by \cite{childs2002}. More modestly, \cite{chakrabortyHow20} shows
% that the upper bound for the speedup of quantum walk \textit{mixing time} is
% \textit{almost} quadratic for \textit{almost} all networks. \par

% The ballistic behavior introduced by quantum interference can be better
% understood by looking closer at localization quantities such the
% \textit{inverse participation ratio}, which gives the average number of nodes
% the wave function is spread over, providing further insight on how the walker
% escapes or maintains its given position. \cite{zengDiscrete-time17} studies
% the relationship between this quantity and the standard deviation when
% phase disorders are introduced into the walk. \cite{buarqueAperiodic19} further
% investigates these transport properties on aperiodic discrete-time quantum
% walks. \par

% Another feature of quantum transport is creating communication channels through
% a phenomenon known as \textit{quantum state transfer}. \cite{boseQuantum03}
% proposes a setup composed of unmodulated and unmeasured spin-chains to transfer
% quantum information in short distances with better fidelities than classical
% channels, achieving \textit{perfect state transfer} (PST) for specific
% configurations. Further work by \cite{christandlPerfect04} discovers a class of
% qubit networks that exhibit PST for any quantum state in a specific time, and
% \cite{godsilWhen11} defines a procedure to investigate if a given adjacency
% matrix of a graph allows PST to occur. In the context of the continuous-time
% quantum walk, \cite{alvirPerfect14} formulates and studies this effect on
% irregular graphs described by Laplacian matrices. This was extended to CTQWs on
% directed and oriented structures in works such as \cite{godsilPerfect2020} and
% \cite{chavesWhy22}, although the behavior of transport properties in these
% structures is still not very well researched, meaning there are possible
% algorithmic advantages to be found.\par 

\subsubsection{Statistical Measures}

Transport properties of CTQWs can be quantified using metrics like mean and
standard deviation, which shed light on the walker's position tendencies and
propagations. The mean is defined as
\begin{equation}
    \mu = \sum_{x} x |\langle \psi | x \rangle|^2
\end{equation}
and the standard deviation is
\begin{equation}
    \sigma = \sqrt{\sum_{i} (x - \mu)^2 |\langle \psi | x \rangle|^2},
\end{equation}
notably, as aforementioned, the standard deviation evolves quadratically faster
than in the classical case \cite{grimmett2003}.

%The transport characteristics of continuous-time quantum walks can be
%quantified through statistical measures such as the mean and standard
%deviation. These properties provide insights into the central tendency and
%dispersion of the walker's positions, serving as valuable indicators of the
%walk's behavior. We define the mean value as
%\begin{equation}
%    \mu = \sum_{x} x |\langle \psi | x \rangle|^2
%\end{equation}
%and the standard deviation as
%\begin{equation}
%    \sigma = \sqrt{\sum_{i} (x - \mu)^2 |\langle \psi | x \rangle|^2},
%\end{equation}
%and we might notice the standard deviation is quadratically faster in
%comparison to its classical counterpart \cite{grimmett2003}.

To quantify the walker's localization, we study the likelihood of finding the
walker at a specific position after time $t$, namely the \textit{survival
probability} \cite{Goenuelol2011}. For a symmetric position range of $ 
[k_0,k_1]$ , it is given by
\begin{equation}
    P_{[k_0,k_1]}(t) = \sum_{i=k_0}^{k_1} |\bra{i}\ket{\psi(t)}|^2.
\end{equation}

%In order to quantify the localization of the quantum walker, we may define the
%survival probability of a quantum walk, which can be characterized as the
%probability of finding the walker in a certain location after some time $t$
%\cite{Goenuelol2011}. Considering the symmetric position range of $[k_0,k_1]$,
%can be written as
%\begin{equation}
%    P_{[k_0,k_1]}(t)=\sum_{i=k_0}^{k_1} |\bra{i}\ket{\psi(t)}|^2.
%\end{equation}

Moreover, the \textit{inverse participation ratio} (IPR) provides an estimate
of the average number of nodes over which the walker is uniformly spread, and
it is defined as
\begin{equation}
    \text{IPR(t)} = \frac{1}{\sum_{x} |\langle x | \psi(t) \rangle|^4},
\end{equation} 
with IPR nearing 1 for uniform spread across positions and approaching 0 when
the walker localizes on a single site as $N$ increases \cite{Danaci2021}.


%Moreover, the inverse participation ratio (IPR) provides an estimate of the
%average number of nodes in a graph where the walker is spread over uniformly,
%and we can define as
%
%\begin{equation}
%    \text{IPR(t)} = \frac{1}{\sum_{x} |\langle x | \psi(t) \rangle|^4},
%\end{equation} 
%
%and IPR approaches 1 when the walker is evenly spread across the position
%space, and IPR tends to 0 as the walker becomes localized on a single site, as
%N approaches infinity \cite{Danaci2021}.

\subsubsection{Perfect State Transfer}

Christandl \textit{et al.} \cite{christandlPerfect04} pioneered the exploration
of state transfers in graphs as a means of information transport. They
questioned whether an initial state at the beginning of a path graph could
relocate to its end after a certain time $t$. Their findings negated this for
undirected graphs, but their work inspired further investigations into which
graphs exhibit this phenomenon and under which conditions. This has since been
explored in both undirected \cite{Coutinho2014, Cheung2014, Zhou2014} and
directed graphs \cite{Lato2020, Cameron2014}.

%One of the ways to transport information in a graph is related to state
%transfers. Christandl \textit{et al.} \cite{christandlPerfect04} was one of the
%first to investigate the phenomenon. They investigated if it was possible for
%an initial state associated to the vertex in the beginning of a path graph to
%be found in the last vertex of the graph after some time $t$. They concluded
%that it was not possible for undirected graphs, but it instigated the
%exploration of which graphs have the phenomenon and under what conditions. It
%has been a research topic for both undirected graphs \cite{Coutinho2014,
%Cheung2014, Zhou2014} and directed graphs \cite{Lato2020, Cameron2014}.

As was previously done in equation \eqref{eq:contSimulUniOp}, given an
Hamiltonian associated to the adjacency matrix of a graph $G$, the time
evolution operator can be defined as $U(t) = e^{-itA}$.
%Define a Hamiltonian associated to the adjacency matrix of a graph $G$, $A$,
%with vertex and edge set $(E,V)$. The time evolution operator in that case will
%be defined as
%
%\begin{equation}
%    U(t) = e^{-itA}.
%\end{equation}

Perfect state transfer (PST) occurs at time $\tau$ between vertices $a$ and $b$
iff
\begin{equation}
    \abs{\bra{a}U(\tau)\ket{b}}^2 = 1,
\end{equation}
such that $\ket{k}$ is the characteristic vector of the vertex $k$ (all entries
of the vector are zero except for the $k$-th position).

PST is not universal across all graphs; the adjacency matrix's eigenvalues and
eigenvectors must obey very specific criteria. The conditions are well
known for the undirected case \cite{Godsil2012}, but a complete description is
still an open problem for oriented graphs \cite{Lato2020}. Although finding a
graph with PST can be quite difficult, checking if the adjacency matrix, for
the undirected case, fits in the criteria can be done in polynomial time
\cite{Coutinho2017}. Examples of PST include cartesian products of graphs and
hypercube graphs \cite{Coutinho2014, Cheung2014, Zhou2014}. Throughout this
work, we'll showcase our algorithmic implementation that determines if a graph
exhibits PST and its corresponding time.



%PST does not occur for every graph and both eigenvalues and eigenvectors of the
%adjacency matrix need to obey very specific rules. The conditions are well
%known for the undirected case \cite{Godsil2012}, but a complete description is
%still an open problem for oriented graphs \cite{Lato2020}. Although finding a
%graph with PST can be quite difficult, checking if the adjacency matrix, for
%the undirected case, fits in the criteria can be done in polynomial time
%\cite{Coutinho2017}. 

%Literature offers numerous examples of graphs with PST, as a few examples we %have:
%
%\begin{itemize}
%    \item[1.] The complete graph with two vertices allow perfect state transfer
%        between its vertices in time $t = \pi/2$.
%
%    \item[2.] The path graph with three vertices, $P_3$, has perfect state
%        transfer between its end vertices in time $t = \pi/ 2$.
%
%    \item[3.] The hypercube of any order has perfect state transfer between its
%        antipodal points in the same time $t =\pi/2$.
%
%    \item[4.] Any order of Cartesian product of $P_3$ has PST between its
%        antipodal vertices in the same time $t = \pi/ \sqrt{2}$.
%        \cite{Christandl2004, Christandl2005}.
%
%\end{itemize}

%There are known examples of PST like cartesian products of graphs with PST or
%hypercubes graphs\cite{Coutinho2014, Cheung2014, Zhou2014} and through out this
%work we will see many examples of PST as they will be used to test the
%algorithm implementation of a function to check if a graph has PST and at what
%time.

%This work has a bunch of references on bipartite doubles and association schemes \cite{Coutinho2015}

\end{document}
