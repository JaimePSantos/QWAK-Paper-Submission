\documentclass[../../main.tex]{subfiles}
\begin{document}

The stochastic CTQW can be viewed as a CTQW with added noise, therefore
requiring the use of density matrices. The system's evolution is then
represented by 
\begin{equation}
    \rho = \sum_j p_j\ket{\psi_j}\bra{\psi_j},
\end{equation}
where $\ket{\psi_j}$ are pure states, and $p_j$ are non-negative
coefficients summing to one. The underlying graph for state $\rho$ is
characterized by an adjacency matrix $A$, where $A_{i,j}=1$ denotes a link
between nodes, and $A_{i,j}=0$ its absence.

%To describe the stochastic CTQW, and we can interpret as a CTQW with added
%noise, we will need to introduce the formalism of density matrices where the
%evolution of the system is described by $\rho=\sum_j
%p_j\ket{\psi_j}\bra{\psi_j}$ for some pure states $\ket{\psi_j}$ and
%coefficients $p_j$ that are non-negative and add up to one.
%The graph on which the state $\rho$ lives, is defined by links between pair of
%nodes, which are described by an adjacency matrix $A$. The coefficient
%$A_{i,j}=1$ indicates the presence of the link, while $A_{i,j}=0$ indicates its
%absence.\par 

This system's evolution follows the quantum stochastic walk model
\cite{whitfield2010, Caruso2014}, governed by the Lindblad master equation
\cite{Lindblad1976}. Here, the Hamiltonian $H$ is given by the 
adjacency matrix, i.e., $H=A$, and Lindblad operators introduce noise to the
quantum walker
%The evolution of the quantum system has been based on the quantum stochastic
%walk model \cite{whitfield2010, Caruso2014} that can be described by a Lindblad
%master equation \cite{Lindblad1976} where the Hamiltonian of the system $H$
%corresponds to the adjacency matrix itself, i.e. $H=A$, and with a set of
%Lindblad operators acting as noise for the quantum walker, i.e.
\begin{equation}
    {\dot \rho} = -(1-p)\ i[H,\rho] + p\  \mathcal{L}_{CRW} (\rho) + \mathcal{L}_{sink} (\rho)
    \label{densityEvolution}
\end{equation}
with a term describing classical random walk (CRW) regime as

%with a term describing (for $p=1$) the regime of a classical random walk (CRW) as
\begin{equation}
    \mathcal{L}_{CRW} (\rho) = \sum_{i,j} L_{ij} \rho L_{ij}^{\dagger} - \frac{1}{2} \{L_{ij}^\dagger L_{ij}, \rho\}.
\end{equation}
The Lindblad operators, $L_{ij}$, are given by $L_{ij} =
(A_{ij}/d_j)\ket{i}\bra{j}$, where $\lbrace d_j \rbrace$ are the node degrees.
When $p=0$, we recover the pure quantum walker, while $p=1$ reflects a
classical random walker. For $p$ within $[0,1]$, the quantum stochastic walker
emerges, blending coherent evolution and the effects of noise. Additionally, a
special Lindblad operator can irreversibly transfer the population from an
"exit" node $n$ to sink $S$ at rate $\Gamma$ as follows

%\begin{equation}
%    \mathcal{L}_{CRW} (\rho) = \sum_{i,j} L_{ij} \rho L_{ij}^{\dagger} - \frac{1}{2} \{L_{ij}^\dagger L_{ij}, \rho\},
%\end{equation}
%with the Lindblad operators being defined as $L_{ij} =
%(A_{ij}/d_j)\ket{i}\bra{j}$, where $\lbrace d_j \rbrace$ are the node degrees.
%When $p=0$ one recovers the pure quantum walker regime, while $p=1$ corresponds
%to the classical scenario for a random walker. For the intermediate values of
%$p$ in the range $[0,1]$, one has a quantum stochastic walker with an interplay
%between coherent evolution and noise effects. In addition, we can have a
%special Lindblad operator that irreversibly transfers the population from an
%"exit" node $n$ to the sink $S$ with a rate $\Gamma$, as follows
%\begin{equation}
%    \mathcal{L}_{sink} (\rho) = \Gamma \left [ 2\ket{S}\bra{n}\rho \ket{n} \bra{S} - \{\ket{n}\bra{n}, \rho\} \right ].
%\end{equation}
%The sink $S$, which irreversibly traps the population constitutes a kind of
%\emph{exit} node and it can be useful to model phenomena such as energy
%transfer. Equation \ref{densityEvolution} gives leads to the following
%expression for the escape probability of the walker trough the sink  

\begin{equation}
    \mathcal{L}_{sink} (\rho) = \Gamma \left [ 2\ket{S}\bra{n}\rho \ket{n} \bra{S} - \{\ket{n}\bra{n}, \rho\} \right ].
\end{equation}
The sink \(S\), acting as an \emph{exit} node, irreversibly captures the
population and can be a useful model for phenomena like energy transfer. From
equation \eqref{densityEvolution}, we derive the escape probability of the walker
through the sink


% \textcolor{red}{
% \begin{itemize}
%     \item We say above that the special operator transfers the population from
%         an exit node to the sink, but here we say the sink node is the exit
%         node. Is this correct?
% \end{itemize}
% }

\begin{equation}
    p_{sink}(t) = 2 \Gamma \int_0^t \rho_{n,n}(t') \ \dd t' \; .
\end{equation}


The stochastic CTQW has numerous applications such has modelling energy
transfer phenomena \cite{Caruso2009}, state discrimination \cite{Pozza2020} and
reinforcement learning \cite{Pozza2022,Buffoni2020} among the others. It also
serves as a useful model that can easily interpolate, via the parameter $p$,
between a purely unitary evolution ($p=0$) and a classical one ($p=1$).


%The Stochastic CTQW introduced here have found numerous applications in
%modelling energy transfer phenomena \cite{Caruso2009}, state
%discrimination \cite{Pozza2020} and reiforcement learning
%\cite{Pozza2022,Buffoni2020} among the others.  They also serve as a
%useful model that can easily interpolate, via the parameter $p$, between a
%purely unitary evolution ($p=0$) and a classical one ($p=1$).

\end{document}
