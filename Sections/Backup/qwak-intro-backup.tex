
\documentclass[../../dissertation.tex]{subfiles}
\begin{document}
In this section, we present an object-oriented Python package for the
realization of unitary continuous-time quantum walks, as well as stochastic
quantum walks, which are generalizations of the unitary case, allowing for
classical effects to take place. Additionally, a graphical user interface
created with \texttt{HTML}, \texttt{CSS} and \texttt{JavaScript} is provided as
a user-friendly alternative, where it is possible to study several different
aspects of the quantum walk without having prior coding knowledge. The user can
either define a custom graph by setting the number of nodes and drawing
connections between them or use pre-defined structures provided by
\texttt{NetworkX}. The front and the backend are linked through the
\texttt{Eel} package, which is a lightweight alternative to \texttt{Electron}.\par

We begin with a description of the \texttt{QWAK} package, explaining the
installation procedure and the dependencies required to use this software,
followed by a high-level description of the implementation and the associated
challenges, as well as instructions for package usage. The
%TODO: And future work parece um mau ponto para terminar.
section is then closed with an overview of the graphical interface and future
work.
 
\end{document}
