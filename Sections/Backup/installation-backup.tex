\documentclass[main.tex]{subfiles}
\begin{document}

The \texttt{QWAK} package can be currently installed by firstly cloning the
\texttt{GitHub} \footnote{https://github.com/JaimePSantos/QWAK} repository:

\begin{lstlisting}[style=commands]
$   git clone https://github.com/JaimePSantos/QWAK.git 
\end{lstlisting}
and, after navigating to the cloned folder, run the following \texttt{pip}
command: 
\begin{lstlisting}[style=commands]
$   pip install .
\end{lstlisting}

The primary purpose of this software is to create an easy-to-use set of tools
that enables the user to simulate the probability distribution and investigate
transport properties of a continuous-time quantum walk performed over a wide
range of graphs. Because we are interested in numerical simulation,
\texttt{NumPy} is an indispensable asset because of its multidimensional
arrays, and high-performing functions over these structures, allowing for
efficient calculations of essential routines such as Fourier transforms,
eigenvalues and eigenvectors. \texttt{SciPy} is also required since it provides
complementary functions such as matrix exponentiation and inverse matrix
calculation. Lastly, the \texttt{QuTiP} package covers the simulation of the
dynamics of open quantum systems, which is crucial for solving, for example,
the master equation of the evolution of a density matrix.\par

\texttt{NetworkX} is responsible for the structures of our walks. This package
contains an extensive library of pre-defined graphs and allows the
creation of custom graphs equipped with weights, helpful in adding direction
and orientation. It is also in charge of calculating the adjacency or Laplacian
matrix, depending on the regularity of the graph.\par

Visualizing probability distributions and features of the quantum walk is one
of our main goals, which is why the final backend requirement is
\texttt{Matplotlib}, a package capable of generating static or animated plots
from the calculated arrays. Alternatively, the user can build and visualize
both static and dynamic plots of the probability distribution and transport
properties in the graphical user interface, requiring the \texttt{Eel} package
to be installed so that Python functions can be called from
JavaScript and vice-versa.\par

In the future, the package will also be hosted on \texttt{PyPi} to
simplify the installation process. Currently, however, the user is advised to
follow the repository instructions. The graphical interface will still have to
be manually installed since it is external to the package.

\end{document}
