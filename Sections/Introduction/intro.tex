\documentclass[main.tex]{subfiles}
\begin{document}

% \begin{itemize}
%     \item Only simulation context is here. We should probably add some other
%         stuff and cut down on the detail of simulation.
%     \item Add a paragraph about quantum computing packages after second paragraph.
%     \item Another paragraph about quantum walks.
%     \item \color{blue}{we need to change the way we cite, for example, 'in the work of [1]'}
% \end{itemize}

%\textcolor{blue}{
The study of quantum systems in a programmable manner has been a topic of
interest since the proposal of the concept of a quantum computer as introduced
by Feynman in his seminal work \cite{feynman1982}. He demonstrated that
classical Turing machines fall short in simulating certain quantum phenomena,
which quantum computers can address. Feynman also emphasized that classical
simulation of quantum systems is often very resource-intesive, while quantum
computers can simulate multi-particle systems with only a corresponding number
of qubits.
%}

%The study of quantum systems in a programmable manner has been a topic of
%interest since the proposal of the concept of a quantum computer as introduced
%by Feynman in his seminal work \cite{feynman1982}. This work showcased that
%conventional classical Turing machines would not be able to simulate certain
%quantum phenomena covered by by the theoretical framework of a quantum
%computer. Feynman also highlights the resource-intensive nature of simulating
%quantum systems on classical computers, while quantum computers could simulate
%a system of many particles requiring only an equivalent number of qubits.\par

%\textcolor{blue}{
After decades of development, quantum computing has evolved from theory to
practice with two major paradigms: \textit{gate-based} and \textit{quantum
annealing} \cite{willschLecture22}. The former has various implementations such as 
\textit{superconducting} (IBM, Google), \textit{photonic} (Xanadu,
PsiQuantum), and \textit{silicon-based} (Intel) quantum computers. The latter,
exemplified by D-Wave, excels in optimization problems but lacks universal computing
capabilities. Currently, we are in the \textit{Noisy Intermediate-Scale
Quantum} (NISQ) era, characterized by small, error-prone qubit arrays. Thus,
classical simulation remains relevant for studying algorithm scalability and
transport properties.
%}

%After several decades since the first idea of quantum computing was born, we've
%seen significant progress being made, showcasing tangible progress from
%theoretical to practical implementation. Currently, there are two major
%paradigms of quantum computing \cite{willschLecture22}. The initial one is
%referred to as \textit{gate-based} quantum computing, and there are several
%versions of it accessible in the market. Examples include
%\textit{superconducting quantum computers} by companies such as IBM and Google,
%\textit{photonic quantum computers} by Xanadu and PsiQuantum, and
%\textit{silicon-based quantum computers} by Intel. The second paradigm is
%\textit{quantum annealing}, or \textit{adiabatic quantum computing}, with an
%implementation by D-Wave. Quantum annealing offers many advantages regarding
%optimization problems. However, it is not suitable for universal quantum
%computing in contrast with the gate-based paradigm. Even though running a
%quantum algorithm on a quantum computer will be the most efficient solution in
%the future, we are currently in the \textit{Noisy Intermediate-Scale quantum}
%(NISQ) era, where quantum hardware is limited to a small number of error-prone
%qubits. For this reason, classical simulation of quantum systems and algorithms
%is still practical. Additionally, it is advantageous to study the scalability
%of algorithms and transport properties in order to gain insights.\par

%\textcolor{blue}{
Our primary objectives include simulating continuous-time quantum walks
(CTQWs), investigating quantum transport properties, and exploring stochastic
quantum walks (SQWs). Early software for quantum walk simulation includes
Marquezino and Portugal's \cite{marquezino2008} general simulator for one- and
two-dimensional lattices. Sawerwain and Gielerak \cite{sawerwain2010} extended
this by leveraging GPU and CUDA technology for enhanced simulations. Berry
\textit{et al.} \cite{berry2011} further contributed with a package that not
only simulates discrete-time quantum walks on arbitrary undirected graphs but
also visualizes time-evolution and supports continuous-time quantum walk
plotting with external data. 
%}

%One of our primary objectives is to simulate continuous- and discrete-time
%quantum walks, along with their algorithms, and investigate the properties of
%quantum transport, as well as the Stochastic Quantum Walk. One of the earliest
%software for the numerical realization of quantum walks include the work of
%Marquezino and Portugal \cite{marquezino2008}, who developed a general
%simulator for discrete-time quantum walks on one- and two-dimensional lattices.
%Sawerwain and Gielerak \cite{sawerwain2010} later presented further work on
%these structures, where they studied the simulation of quantum walks by taking
%advantage of the GPU and CUDA technology. Another interesting program for
%simulating discrete-time quantum walks came with the work of Berry, Bourke, and
%Wang \cite{berry2011}. This package enables direct simulation of these quantum
%walks and visualization of the time-evolution on arbitrary undirected graphs.
%It also offers the functionality to create plots for continuous-time quantum
%walks with externally provided data.\par 


%\textcolor{blue}{
Direct simulation tools for CTQWs are well documented in the literature, with
early work by Izaac and Wang \cite{izaac2015} introducing efficient simulation
software for multi-particle systems on \textit{High-Performance Computing}
platforms, albeit developed in \texttt{Python 2} and no longer supported.
Fallon \textit{et al.} \cite{falloon2017a} provide a \textit{Mathematica}
package for SQWs, incorporating both coherent and incoherent dynamics, thus
enabling simulations of quantum and classical random walks. This approach has
facilitated diverse applications, from modeling energy capture in
photosynthetic complexes \cite{mohseni08} to developing page ranking algorithms
\cite{SanchezBurillo2012}. Glos \textit{et al.} \cite{glos2018}
later ported this package to the \texttt{Julia} language. \texttt{Qutip}
\cite{Johansson2011,Johansson2012} is an alternative \texttt{Python} solver for
these kinds of systems, and is integrated into our package for SQW simulation.
%}

%Direct simulation tools for continuous-time quantum walks can be observerd in
%existing literature, with one of the earliest examples attributed to Izaac and
%Wang \cite{izaac2015}. Their distributed memory software claims to perform
%efficient simulation of multi-particle continuous-time quantum walk-based
%systems on \textit{High-Performance Computing} platforms. Unfortunately, this
%program was written in \texttt{Python 2}, and the creators no longer support it.
%Fallon, Rodriguez, and Wang \cite{falloon2017a} provide a \textit{Mathematica}
%package that implements a simulator of \textit{Quantum Stochastic Walks},
%extending the continuous-time model to a broader context. These walks
%incorporate both coherent and incoherent dynamics, meaning that quantum
%stochastic walks can be instantiated as quantum walks and classical random
%walks. This paper then provides a way of implementing quantum walks on many
%structures, opening the door to applications ranging from modeling the capture
%of energy by photosynthetic protein complexes, as shown by Mohseni et al.
%\cite{mohseni08}, to page ranking algorithms used by search engines. Glos,
%Miszczak, and Ostaszewski \cite{glos2018} subsequently ported and enhanced this
%package to the Julia programming language.\par

Given the current landscape, there's a clear demand for updated, user-friendly
simulation software for CTQWs, especially one that can analyze evolution
on generalized graphs with arbitrary direction, weight, and orientation.
Additionally, the software is capable of exploring quantum walk-based
algorithms and transport properties. While some tools are availale, none we've
encountered fully meet the requirements for highly general graphs. 

%Given this context, it becomes apparent that an updated, user-friendly
%simulation software Continuous-Time Quantum Walks (CTQWs) is in demand,
%particularly one equipped to analyze the evolution of the walk on a generalized
%graph of arbitrary direction, weight and orientation. Moreover, a software that
%has the capacity to explore quantum walk-based algorithms and transport
%properties. It is worth highlighting that, as far as our knowledge extends,
%there is currently no existing package capable of fulfilling these requirements
%for highly general graphs.\par

%\textcolor{blue}{
The paper is organized as follows: Section \ref{sec:theor_background} lays out
the theoretical foundation, elucidating continuous-time quantum walks, their
stochastic variation, and relation to transport properties. Section
\ref{sec:pack_overview} presents a detailed package overview, including an
installation guide, a tutorial, and a description of the graphical interface.
Section \ref{sec:use_cases} showcases the package's versatility through several 
use cases. We wrap up with Section \ref{sec:conclusion}, offering conclusions
and insights into future research directions.
%}

%In the following sections, we will explore the core content of this paper.
%Section 2 will provide the necessary theoretical foundation, explaining the
%basic concepts of continuous-time quantum walks, as well as the stochastic
%variation, and how they relate to transport properties. In Section 3, we will
%shift our focus to practical matters, offering a comprehensive package
%overview, step-by-step installation instructions, and a user-friendly guide for
%basic usage, including a graphical interface. Section 4 will explore use cases,
%highlighting the versatility of our package in various scenarios. Finally, in
%Section 5, we will conclude our discussion and provide insights into the
%broader implications and future directions of our research.

\end{document}
