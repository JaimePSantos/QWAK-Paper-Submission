\documentclass[../../main.tex]{subfiles}
\begin{document}

%\begin{itemize}
%    \item Small introduction for example usage of PST using qwak.
%    \item small setup for the code in the next block.
%\end{itemize}
%
%\begin{lstlisting}[style=code,escapeinside={__}]
%n = 4
%graph = nx.hypercube\textunderscore_graph(n)
%initcond = [0]
%
%qw = QWAK(graph=graph)
%t = eval(str(qw.checkPST(0,15)))
%qw.runWalk(time=t, initStateList=initcond)
%plt.plot(qw.getProbVec())
%\end{lstlisting}
%
%\begin{itemize}
%    \item Quick code overview. 
%    \item Quick setup for the results in the next block. 
%\end{itemize}
%
%
%\begin{figure}[!h]
%    \centering
%    \includegraphics[width=12cm]{img/QWAK/PerfectStateTransfer/Hypercube_N16_T6.28_FROM0_TO15.png}
%    \caption{Directed infinite line with weights $e^{i\alpha}$ and $e^{-i\alpha}$.}
%    \label{fig:oriented_line}
%\end{figure}
%
%\begin{itemize}
%    \item Short analysis of results. 
%    \item Conclusion for this subsection and setup for the next. 
%\end{itemize}

To showcase the capabilities of the perfect state transfer (PST) functionality
within QWAK, we present a hands-on example featuring a quantum walk on a
4-dimensional hypercube graph. This choice of graph is particularly instructive
because hypercube graphs are a well-studied class of graphs with interesting
topological properties. For this demonstration, we initialize the quantum walk
at vertex $0$. This choice of initial vertex serves as the starting point for
the state transfer process. The algorithm was based on the work of Coutinho \textit{et al.}\cite{coutinho17}.

\begin{lstlisting}[style=code,escapeinside={__}]
n = 4
graph = nx.hypercube_\textunderscore_graph(n)
initcond = [0]

qw = QWAK(graph=graph)
t = eval(str(qw.checkPST(0,15)))
qw.runWalk(time=t, initStateList=initcond)
plt.plot(qw.getProbVec())
\end{lstlisting}

In the above code, a 4-dimensional hypercube with 16 nodes is created using
\texttt{NetworkX}. The walker starts at vertex 0, and the \texttt{checPST}
function returns the time value for which PST occurs between the antipodal
vertices, or $-1$ if it does not. For this hypercube, PST happens at $t =
\frac{\pi}{2}$, consistent with all $N$-dimensional hypercubes
\cite{christandlPerfect04,Christandl2005}. This can be confirmed by iteratively
running the \texttt{checkPST} function for all antipodal vertex pairs contained
in the structure. The quantum walk is then executed for this time duration via
\texttt{runWalk}, and \texttt{Matplotlib} is used to plot the resulting
probability vector, showing complete transfer from vertex $0$ to $15$.\par

\begin{figure}[!h]
    \centering
    \includegraphics[scale=\mysinglefigurescale]{img/QWAK/PerfectStateTransfer/Hypercube_N16_T6.28_FROM0_TO15.png}
    \caption{Perfect state transfer between antipodal nodes $0$ and $15$, for a hypercube of size $N=16$.}
    \label{fig:hypercube_pst}
\end{figure}

Figure \ref{fig:hypercube_pst} was plotted using an utility function present in
the notebook, illustrating the evolution of the probability of finding a walker
between the vertices we expect PST to occur. PST is a cyclic phenomenon that
occurs every $2\tau$, being $\tau$ the initial PST time (in this case $\pi/2$).
This is a direct consequence of the fact that there is PST from vertex $a$ to
$b$, then there is also from $b$ to $a$. This can all be observed in the figure
with the blue lines indicating the evolution for the initil vertex and the
green line for the final vertex. Notice that PST return to the original vertex
every $t=\pi$, since PST occured at time $\pi/2$ from vertex 0 to vertex 15,
then at time $\pi/2 + \pi$ the walker returns to vertex 15. PST is successfully
achieved, confirming the efficacy of the algorithm.

This example serves as a practical validation of the theoretical framework
discussed earlier and sets the stage for more complex simulations and analyses
in the following sections.


\end{document}
