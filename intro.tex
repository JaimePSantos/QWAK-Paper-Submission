\documentclass[main.tex]{subfiles}
\begin{document}

The study of quantum systems in a programmable manner has been a topic of
interest since the proposal of the concept of a quantum computer as introduced
by Feynman in his seminal work \cite{feynman1982}. He demonstrated that
classical Turing machines fall short in simulating certain quantum phenomena,
which quantum computers can address. Feynman also emphasized that classical
simulation of quantum systems is often very resource-intesive, while quantum
computers can simulate multi-particle systems with only a corresponding number
of qubits.

After decades of development, quantum computing has evolved from theory to
practice with two major paradigms: \textit{gate-based} and \textit{quantum
annealing} \cite{willschLecture22}. The former has various implementations such as 
\textit{superconducting} (IBM, Google), \textit{photonic} (Xanadu,
PsiQuantum), and \textit{silicon-based} (Intel) quantum computers. The latter,
exemplified by D-Wave, excels in optimization problems but lacks universal computing
capabilities. Currently, we are in the \textit{Noisy Intermediate-Scale
Quantum} (NISQ) era, characterized by small, error-prone qubit arrays. Thus,
classical simulation remains relevant for studying algorithm scalability and
transport properties.

Our primary objectives include simulating continuous-time quantum walks
(CTQWs), investigating quantum transport properties, and exploring stochastic
quantum walks (SQWs). Early software for quantum walk simulation includes
Marquezino and Portugal's \cite{marquezino2008} general simulator for one- and
two-dimensional lattices. Sawerwain and Gielerak \cite{sawerwain2010} extended
this by leveraging GPU and CUDA technology for enhanced simulations. Berry
\textit{et al.} \cite{berry2011} further contributed with a package that not
only simulates discrete-time quantum walks on arbitrary undirected graphs but
also visualizes time-evolution and supports continuous-time quantum walk
plotting with external data. 

Direct simulation tools for CTQWs are well documented in the literature, with
early work by Izaac and Wang \cite{izaac2015} introducing efficient simulation
software for multi-particle systems on \textit{High-Performance Computing}
platforms, albeit developed in \texttt{Python 2} and no longer supported.
Fallon \textit{et al.} \cite{falloon2017a} provide a \textit{Mathematica}
package for SQWs, incorporating both coherent and incoherent dynamics, thus
enabling simulations of quantum and classical random walks. This approach has
facilitated diverse applications, from modeling energy capture in
photosynthetic complexes \cite{mohseni08} to developing page ranking algorithms
\cite{SanchezBurillo2012}. Glos \textit{et al.} \cite{glos2018}
later ported this package to the \texttt{Julia} language. \texttt{Qutip}
\cite{Johansson2011,Johansson2012} is an alternative \texttt{Python} solver for
these kinds of systems, and is integrated into our package for SQW simulation.

Given the current landscape, there's a clear demand for updated, user-friendly
simulation software for CTQWs, especially one that can analyze evolution
on generalized graphs with arbitrary direction, weight, and orientation.
Additionally, the software is capable of exploring quantum walk-based
algorithms and transport properties. While some tools are availale, none we've
encountered fully meet the requirements for highly general graphs. 

The paper is organized as follows: Section \ref{sec:theor_background} lays out
the theoretical foundation, elucidating continuous-time quantum walks, their
stochastic variation, and relation to transport properties. Section
\ref{sec:pack_overview} presents a detailed package overview, including an
installation guide, a tutorial, and a description of the graphical interface.
Section \ref{sec:use_cases} showcases the package's versatility through several 
use cases. We wrap up with Section \ref{sec:conclusion}, offering conclusions
and insights into future research directions.

\end{document}
